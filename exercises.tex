%==============================================================================
%  Algebraic Topology - Exercises
%  Initial LaTeX Commit
%==============================================================================

\documentclass[12pt]{book}

%------------------------------------------------------------------------------
%   Packages
%------------------------------------------------------------------------------
\usepackage[utf8]{inputenc}
\usepackage[T1]{fontenc}
\usepackage{lmodern}

\usepackage{tikz}
\usepackage{tikz-cd}
\usepackage{quiver}
\usetikzlibrary{calc,3d}
\usetikzlibrary{decorations.markings}
\usepackage{animate}

\usepackage{amsmath,amssymb,amsthm}
\usepackage{mathtools}    % For extra math features (e.g., \DeclareMathOperator)
\usepackage{geometry}
\geometry{margin=1in}

\usepackage{xparse}
\usepackage{xcolor}

\usepackage{dsfont}


%------------------------------------------------------------------------------
%   Custom Commands for Algebraic Topology
%------------------------------------------------------------------------------
% Reduced homology, e.g. \rH_n(X)
\newcommand{\rH}{\widetilde{H}}

% Common functors in algebraic topology
\DeclareMathOperator{\Hom}{Hom}
\DeclareMathOperator{\Ext}{Ext}
\DeclareMathOperator{\Tor}{Tor}
\DeclareMathOperator{\colim}{colim}
\DeclareMathOperator{\hocolim}{hocolim}
\DeclareMathOperator{\holim}{holim}

% Smash and wedge products, etc.
\newcommand{\wedgeprod}{\wedge}
\newcommand{\wedgesum}{\vee}

% Often we override \Im, \Re, \ker, etc. for nicer formatting:
\renewcommand{\Im}{\operatorname{Im}}
\renewcommand{\Re}{\operatorname{Re}}
\DeclareMathOperator{\Ker}{Ker}

% Common Fields
\newcommand{\R}{\mathbb{R}}        % Real numbers
\newcommand{\C}{\mathbb{C}}        % Complex numbers
\newcommand{\HH}{\mathbb{H}}        % Quarternions
\newcommand{\Q}{\mathbb{Q}}        % Rational numbers
\newcommand{\Z}{\mathbb{Z}}        % Integers
\newcommand{\N}{\mathbb{N}}        % Natural numbers
\newcommand{\K}{\mathbb{K}}        % Arbitrary field

% Homotopic
\newcommand{\heq}{\simeq}
\newcommand{\id}{\mathds{1}}

% Restriction
\newcommand{\restr}[2]{\left.#1\right|_{#2}}


%------------------------------------------------------------------------------
%   Define 'taggedexercise' environment
%------------------------------------------------------------------------------
% We want each exercise to appear in the ToC at subsection level
\NewDocumentEnvironment{taggedexercise}{O{}}
{%
  \refstepcounter{exercise}%
  \par\noindent
  \textbf{Exercise \theexercise}%
  \if\relax\detokenize{#1}\relax
  \else
    \textbf{\ [#1]}%
  \fi
  \quad
  \addcontentsline{toc}{subsection}{%
    Exercise \theexercise
    \if\relax\detokenize{#1}\relax
    \else
      \ [#1]%
    \fi
  }
}{%
  \par
}

%------------------------------------------------------------------------------
%   Theorem-Like Environments
%------------------------------------------------------------------------------
\theoremstyle{definition}
\newtheorem{exercise}{Exercise}[chapter]

% Custom solution environment
\newenvironment{solution}
{%
  \par\noindent\textbf{Solution.}\quad
}
{%
  \qed\par
}
{
  \vspace{2pt}
}

%------------------------------------------------------------------------------
%   Metadata
%------------------------------------------------------------------------------
\title{Exercises from \textit{Algebraic Topology} \\
       by Allen Hatcher}
\author{Tyler Jensen | tyjensen222@gmail.com}
\date{\today}

%------------------------------------------------------------------------------
%   Document
%------------------------------------------------------------------------------
\begin{document}

\frontmatter
\maketitle
\tableofcontents

\mainmatter

%------------------------------------------------------------------------------
%  START CHAPTERS AT 0
%------------------------------------------------------------------------------
\setcounter{chapter}{-1}

%==============================================================================
\chapter{Some Underlying Geometric Notions}
%==============================================================================


% Exercise 0.1
\begin{taggedexercise}[\textcolor{green}{Complete}]
  Construct an explicit deformation retraction of the torus with one point deleted onto a graph of two circles intersecting at a point, namely, longitude and meridian circles of the torus.
\end{taggedexercise}
\begin{solution}
  Consider the fundamental square of the torus with a point removed:
  \begin{center}
    \begin{tikzpicture}[
      scale=2,  % (1) Scale up
      arrow inside/.style = {
        postaction={decorate},
        decoration={markings, mark=at position 0.5 with {\arrow{stealth}}}
      }
    ]
    
      %--- Draw the square with edges labeled a and b (with arrows in the middle)
      \draw[arrow inside] (0,0) -- node [below] {$a$} (1,0);
      \draw[arrow inside] (0,1) -- node [above] {$a$} (1,1);
      \draw[arrow inside] (0,0) -- node [left]  {$b$} (0,1);
      \draw[arrow inside] (1,0) -- node [right] {$b$} (1,1);
    
      %--- (2) Small circle in the center to mark the removed point
      \draw (0.5,0.5) circle (0.03);
    \end{tikzpicture}
  \end{center}
  where the deformation retract of the removed point retaracts to the boundary of the square; identifying the sides gives a bouquet of two circles:
  \begin{center}
    \begin{tikzpicture}[scale=2.0]

      % We define a style "side arrow" that places
      % a large arrow tip at a specified position around the circle.
      % - 'mark=at position <number>' controls where along the path [0..1].
      % - 'scale=1.8' (for example) enlarges the arrowhead.
      \tikzset{
        side arrow/.style={
          postaction={decorate},
          decoration={
            markings,
            mark=at position #1 with {\arrow[scale=1.8]{stealth}}
          }
        }
      }
    
      % Circle 'a'
      %  -- Center: (-0.5,0), Radius: 0.5
      %  -- Place arrow at 0.25 (1/4 around the circle),
      %     so the arrow is off to the side.
      \draw[side arrow=0.25]
           (-0.5,0) circle (0.5);
      % Label 'a' near top
      \node at (-0.5,0.3) {$a$};
    
      % Circle 'b'
      %  -- Center: (0.5,0), Radius: 0.5
      %  -- Put arrow at 0.75 to shift it to the opposite side
      \draw[side arrow=0.75]
           (0.5,0) circle (0.5);
      % Label 'b' near top
      \node at (0.5,-0.3) {$b$};
    
      % Wedge point
      \fill (0,0) circle (0.02);
    
    \end{tikzpicture}
  \end{center}
  where the point at the intersection of the circles is the identification of the four corners.

  Alternatively, let $p \in T$ be the removed point.
  Choose your favorite open set $U$ about $p$.
  Retract $U$ to the meridian, so that $T$ is now missing a longitudinal strip, contracted to  a line spanning the meridial width of $U$.
  Then there are two open copies of $S^1$ seperated by this width; contract them along the meridian in the opposite direction of eachother.
  Then, exactly two copies of $S^1$ are left, connected by a single point.

\end{solution}

% Exercise 0.2
\begin{taggedexercise}[\textcolor{green}{Complete}]
  Construct an explicit deformation retract from $\R^n - \{0\}$ to $S^{n-1}$.
\end{taggedexercise}
\begin{solution}
  Define a map 
  \[h: \left(\R^n - \{0\}\right) \times I \longrightarrow S^{n-1} \subset \R^n - \{0\}\]
  via 
  \[h(x, t):= (1-t)x + \frac{tx}{||x||}\]
  where $||\cdot ||$ is the Euclidean norm in $\R^n$.
  Then, $h(x, 0) = x$ and $h(1) = x/||x||$, retracting each point to a point on $S^{n-1}$.

\end{solution}

% Exercise 0.3
\begin{taggedexercise}[\textcolor{green}{Complete}]
  \begin{enumerate}
    \item Show that the composition of homotopy equivalences $X \to Y$ and $Y \to Z$ is a homotopy equivalence $X \to Z$.
          Deduce that homotopy equivalence is an equivalence relation.
    \item Show that the relation of homotopy among maps $X \to Y$ is an equivalence relation.
    \item Show that a map homotopic to a homotopy equivalence is an equivalence relation.
  \end{enumerate}
\end{taggedexercise}

\begin{solution}
  (1). Define homotopies $f : X \to Y$, $f' : Y \to X$, $g : Y \to Z$ and $g' : Z \to Y$ s.t.
  \[f' \circ f \heq \id_X, f \circ f' \heq \id_Y\]
  \[g' \circ g \heq \id_Y, g \circ g' \heq \id_Z.\]
  Then,
  \[
  \begin{aligned}
    (f' \circ g')\circ (g \circ f) &= f' \circ (g' \circ g) \circ f\\
                                   &= f' \circ \id_Y \circ f \\
                                   &= f' \circ f \\
                                   &= \id_X
  \end{aligned}
  \]
  and 
  \[
  \begin{aligned}
    (g \circ f) \circ (f' \circ g') &= g \circ (f \circ f') \circ g' \\
                                    &= g \circ \id_Y \circ g' \\
                                    &= g \circ g' \\
                                    &= \id_Z
  \end{aligned}
  \]
  hence, the composition map induced by $X \to Y \to Z$ induces a homotopy equivalence $X \to Z$.
  Or, in other words, the following diagram commutes:
  \[\begin{tikzcd}
	&&&& Y \\
	\\
	\\
	X &&&&&&&& Z
	\arrow["{\id_Y}", from=1-5, to=1-5, loop, in=55, out=125, distance=10mm]
	\arrow["{f'}", curve={height=-12pt}, from=1-5, to=4-1]
	\arrow["g", curve={height=-12pt}, from=1-5, to=4-9]
	\arrow["f", curve={height=-12pt}, from=4-1, to=1-5]
	\arrow["{\id_X}", from=4-1, to=4-1, loop, in=145, out=215, distance=10mm]
	\arrow["{g \circ f}"', curve={height=-12pt}, from=4-1, to=4-9]
	\arrow["{g'}"', curve={height=-12pt}, from=4-9, to=1-5]
	\arrow["{f' \circ g'}", shift left=3, curve={height=-12pt}, from=4-9, to=4-1]
	\arrow["{\id_Z}", from=4-9, to=4-9, loop, in=325, out=35, distance=10mm]
\end{tikzcd}.\]
  Thus, homotopy equivalence is transitive.

  Next, let $h: X \to X$ be a homotopy s.t. $h_t = \id_X$ for all $t \in I$.
  Then $X$ is certainly homotopy equivalent to itself, and so homotopy equivalence is reflexive. 

  Lastly, let $k: X \to Y$ be a homotopy.
  Then there exists some $k'$ s.t. $k' \circ k \heq \id_X$ and $k \circ k' \heq \id_Y$.
  Thus, $k'$ induces a homotopy equivalence $Y \to X$, and so homotopy equivalence is symmetric. 
  Therefore, homotopy equivalence is an equivalence relation. 

  (2). Let $f_0,f_1,f_2: X \to Y$ be maps so that $f_0 \heq f_1$ and $f_1 \heq f_2$. 
  Then there exist homotopies $F, G: X \times I \to Y$ s.t. 
  \[F(x, 0) = f_0(x),\; F(x, 1) = f_1(x)\]
  \[G(x, 0) = f_1(x),\; G(x, 1) = f_2(x).\]
  Then, define a map $H: X \times I \to Y$ s.t.
  \[
    H(x,t) :=\;
    \begin{cases}
      F(x,2t),      & 0 \le t \le \tfrac{1}{2},\\[6pt]
      G(x,2t - 1),  & \tfrac{1}{2} \le t \le 1.
    \end{cases}
  \]
  Then certainly $H$ is a homotopy, so $f_0 \heq f_2$.
  Therefore, homotopy relations of maps is transitive.

  Next, let $f: X \to X$ be a map.
  Then, consider the homotopy $F: X \times I \to X$ s.t. $F(x,t) = f(x)$ for all $t$.
  Thus $f \heq f$, and so homotopy relations of maps is reflexive.

  Lastly, let $f_0, f_1: X \to Y$ be homotopic.
  Then there exists a homotopy $F: X \times I \to Y$ s.t. 
  \[F(x, 0) = f_0(x),\; F(x, 1) = f_1(x).\]
  Consider the map 
  \[H(x, t) := F(x, 1 - t)\]
  then,
  \[H(x, 0) = f_1(x),\; H(x, 1) = f_0(x)\]
  and so $H$ is a homotopy from $f_1$ to $f_0$, hence $f_0 \heq f_1 \iff f_1 \heq f_0$.
  Therefore, homotopy relations of maps is symmetric, and homotopy relations of maps is an equivalence relation.

  (3). Let $X \to Y$ be a homotopy equivalence. 
  Then there exist maps $f: X \to Y$ and $g: Y \to X$ s.t. $f \circ g \heq \id_Y$ and $g \circ f \heq id_X$.
  Let $h \heq g \circ f$ be a map homotopic to the homotopy equivalence. 
  Then by (1) and (2), $h$ is an equivalence relation.

\end{solution}

% Exercise 0.4
\begin{taggedexercise}[\textcolor{yellow}{WIP}]
  A \textbf{deformation retraction in the weak sense} of a space $X$ to a subspace $A$ is a homotopy 
  $f_t: X \to X$ such that $f_0 \heq \id_X$, $f_1(X) \subset A$, and $f_t(A) \subset A$ for all $t$.
  Show that if $X$ deformation retracts to $A$ in this weak sense, then the inclusion $A \hookrightarrow X$
  is a homotopy equivalence.
\end{taggedexercise}

\begin{solution}
  Let $i: A \hookrightarrow X$ be such an inclusion.
  % Define a a map $F: X \times I \to X$ via $F(x, t) := f_t(x)$, then 
  % \[F(x, 0 ) = f_0 \heq \id_X, \;\; F(x, 1) = r(x)\]
  Then, note that  
  \[(f_t \circ i)_{t=0} = \restr{f_0}{A} \heq \restr{\id_X}{A} = \id_A\]
  Next, let $f_1 = r$ be the resulting deformation retract in the weak sense, so that $r(X) \subset A$.
  Then, 
  \[(i \circ f_t)_{t=1} = i \circ r = \restr{i}{r(X)} = \id_{r(X)}\]
  where the final equality holds given $r(X) \subset A$.
  Finally, its clear that $A$ must deformation retract to $r(X)$.
  Hence, $i$ is a homotopy equivalence.
\end{solution}

% Exercise 0.5
\begin{taggedexercise}[\textcolor{red}{TODO}]
  
\end{taggedexercise}

% Exercise 0.6
\begin{taggedexercise}[\textcolor{red}{TODO}]
  
\end{taggedexercise}

% Exercise 0.7
\begin{taggedexercise}[\textcolor{red}{TODO}]
  
\end{taggedexercise}

% Exercise 0.8
\begin{taggedexercise}[\textcolor{red}{TODO}]
  
\end{taggedexercise}

% Exercise 0.9
\begin{taggedexercise}[\textcolor{red}{TODO}]
  
\end{taggedexercise}

% Exercise 0.10
\begin{taggedexercise}[\textcolor{red}{TODO}]
  
\end{taggedexercise}

% Exercise 0.11
\begin{taggedexercise}[\textcolor{yellow}{WIP}]
  Show that $f: X \to Y$ is a hopomotopy equivalence if there exist maps $g,h: Y \to X$ such that $f \circ g \heq \id_Y$ 
  and $h \circ f \heq \id_X$.
  More generally, show that $f$ is a homotopy equivalence if $f \circ g$ and $h \circ f$ are homotopy equivalences.
\end{taggedexercise}

\begin{solution}
  Note that 
  \[g \heq (h \circ f) \circ g = h \circ (f \circ g) \heq h.\]
  Therefore, both $h$ and $g$ serve as a two-sided homotopic inverse of $f$, as in 
  \[g\circ f \heq \id_X, \;\; f\circ g \heq \id_Y\]
  \[h\circ f \heq \id_X, \;\; f\circ h \heq \id_Y\]
  Choosing either shows that $f$ is a homotopy equivalence.
  More generally, let $f \circ g$ and $h \circ f$ be homotopy equivalences.
  Then there exist maps $u: X \to X$ and $v: Y\to Y$ s.t. 
  \[(h \circ f) \circ u \heq u \circ (h \circ f) \heq \id_X\]
  \[(f \circ g) \circ v \heq v \circ (f \circ g) \heq \id_Y\]
  Then,
  \[g \circ u \circ h \circ f\]
\end{solution}

% Exercise 0.12
\begin{taggedexercise}[\textcolor{red}{TODO}]
  
\end{taggedexercise}

%==============================================================================
\chapter{The Fundamental Group}
%==============================================================================

% Exercise 1.1
\begin{taggedexercise}[\textcolor{green}{Complete}]
  Show that composition of paths satisfies the following cancellation property:
  if $f_0 \cdot g_0 \heq f_1 \cdot g_1$ and $g_0 \heq g_1$, then $f_0 \heq f_1$.
\end{taggedexercise}
\begin{solution}
  Consider two homotopies $H_1, H_2 : I \times I \to X$ s.t. 
  \[H_1(s, 0) = (f_0 \cdot g_0)(s), \;\;\; H_1(s, 1) = (f_1 \cdot g_1)(s)\]
  \[H_2(s, 0) = g_0(s), \;\;\; H_2(s, 1) = g_1(s).\]
  Then, consider 
  \[H^{*}_2(s,t) = f_0 \cdot H_2(s,t)\]
  which is clearly also a homotopy given by
  \[H^*_2(s, 0 ) = f_0 \cdot g_0, \;\;\; H^*_2(s, 1) = f_0 \cdot g_1.\]
  Therefore, by transitivity, $f_1 \cdot g_1 \heq f_0 \cdot g_1$, where cancelling $g_1$ on the right gives $f_0 \heq f_1$.

\end{solution}

% Exercise 1.2
\begin{taggedexercise}[\textcolor{green}{Complete}]
  Show that the change-of-basepoint homomoprhism $\beta_h$ depends only on the homotopy class of $h$.
\end{taggedexercise}
\begin{solution}

  Let $h_1, h_2 \in [h]$ with $h_1(0) = h_2(0)=x_1$ and $h_1(1) = h_2(1) = x_0$.
  Then consider the maps
  \[\beta_{h_1}, \beta_{h_2} : \pi_1(X, x_1) \to \pi_1(X, x_0).\]
  We want to show that for any $[f] \in \pi_1(X, x_1)$, we have that
  \[\beta_{h_1}([f]) = [h_1 \cdot f \cdot \overline{h_1}] = [h_2 \cdot f \cdot \overline{h_2}] = \beta_{h_2}([f]).\]
  This is equivalent to showing that $h_1 \cdot f \cdot \overline{h_1} \heq h_2 \cdot f \cdot \overline{h_2}$ for any $f \in [f]$.
  Since $\overline{h_1} \heq \overline{h_2}$ and $h_1 \heq h_2$, this follows immediately from \textbf{Exercise 1.1}
  (while additionally using a similar argument for $g_0 \cdot f_0 \heq g_1 \cdot f_1$, $g_0 \heq g_1 \implies f_0 \heq f_1$), 
  hence the result.

\end{solution}

% Exercise 1.3
\begin{taggedexercise}[\textcolor{green}{Complete}]
  For a path-connected space $X$, show that $\pi_1(X)$ is abelian if and only if all change-of-basepoint homomorphisms $\beta_h$
  depend only on the endpoints of $h$.
\end{taggedexercise}

\begin{solution}

  ($\impliedby$) Assume $\pi_1(X)$ is not abelian.
  Then, there exist $[f], [g] \in \pi_1(X)$ with $[g] \cdot [f] \cdot [g]^{-1} \neq [f]$, and thus
  \[
  \beta_g([f]) = [g\cdot f \cdot \bar{g}] = [g] \cdot [f] \cdot [g]^{-1} \neq [f]
  \]
  where the second equality holds due to \textbf{Exercise 1.2}.
  Let $\{x_0\}$ be the constant loop, then
  \[
    \beta_{\{x_0\}}([f]) = [\{x_0\}] \cdot [f] \cdot [\{x_0\}] = [f]
  \]
  and so $\beta_g \neq \beta_{\{x_0\}}$, a contradiciton.


  ($\implies$) Let $\pi_1(X)$ be abelian.
  Let $h_1, h_2 \in [h]$. 
  Then,
  \[\beta_{h_1}([f]) = [h_1 \cdot f \cdot \bar{h_1}] = [h_1 \cdot f \cdot \bar{h_2}] \cdot [h_2 \cdot \bar{h_1}]\]
  where the last equality holds since $[h_2 \cdot \bar{h}]$ is trivial as they are in the same homotopy class.
  Since $\pi_1(X)$ is abelian, we get that 
  \[\beta_{h_1}([f]) = [h_1 \cdot f \cdot \bar{h_2}] \cdot [h_2 \cdot \bar{h_1}] = [h_2 \cdot \bar{h_1}] \cdot[h_1 \cdot f \cdot \bar{h_2}] = [h_2 \cdot f \cdot \bar{h_2}] = \beta_{h_2}([f])\]
  hence the result.

\end{solution}

% Exercise 1.4
\begin{taggedexercise}[\textcolor{red}{TODO}]
  
\end{taggedexercise}

% Exercise 1.5
\begin{taggedexercise}[\textcolor{green}{Complete}]
  Show the following are equivalent:
  \begin{enumerate}
    \item Every map $S^1 \to X$ is homotopic to a constant map, with image a point.
    \item Every map $S^1 \to X$ extends to a map $D^2 \to X$.
    \item $\pi_1(X, x_0) = 0$ for all $x_0\in X$.
  \end{enumerate}
  Deduce that a space is simply-connected iff all maps $S^1 \to X$ are homotopic.
\end{taggedexercise}

\begin{solution}

  (1 $\implies$ 2). Let $f: S^1 \to X$.
  Let $H : S^1 \times I \to X$ be a homotopy from (1) s.t.
  \[H(x, 0) = f(x), \;\; H(s, 1) = \{x_0\}\]
  with the restraint that $H(s, t) \subset f(S^1)$ for all $t > 0$.
  Embed $S^1 \times I$ into $D^2$ as an annulus with center $c$; extend $H$ over all of $D^2$ by mapping it to $c$.

  (2 $\implies$ 3). Think of $\pi_1(X, x_0)$ as basepoint-preserving homotopy classes of maps $f: (S^1, s_0) \to (X, x_0)$.
  By hypothesis $f$ extends to a map $\tilde{f}: (D^2, s_0) \to X$, and so there is a deformation retraction $r$ of $D^2$ onto $s_0$.
  Let $H: S^1 \to X$ be a homotopy s.t. 
  \[H(s, t) := f\circ r_t(s)\]
  which preserves the basepoint $s_0$, and is a homotopy of $f$ to $\{f(s_0)\} = \{x_0\}$, hence the result.

  (3 $\implies$ 1). Again, think of $\pi_1(X, x_0)$ as basepoint-preserving homotopy classes of maps $f: (S^1, s_0) \to (X, x_0)$.
  By hypothesis, there must exist a homotopy taking $f$ to $\{f(s_0)\} = \{x_0\}$, hence the result.

  If all maps $S^1 \to X$ are homotopic, then $\pi_1(X, x_0)$ has exactly one element, which must be the identity, and 
  so $X$ must be simply-connected. 
  Conversely, if $X$ is simply-connected, then by the above, all maps $S^1 \to X$ are homotopic to a constant map 
  at some basepoint.

\end{solution}


% Exercise 1.6
\begin{taggedexercise}[\textcolor{red}{TODO}]

  We can regard $\pi_1(X, x_0)$ as all basepoint-preserving homotopy classes of maps $(S^1, s_0) \to (X, x_0)$.
  Let $[S^1, X]$ be the set of homotopy classes of maps $S^1 \to X$, with no condition on basepoints.
  Thus there is a natural map $\Phi : \pi_1(X, x_0) \to [S^1, X]$ obtained by ignoring basepoints.
  Show that $\Phi$ is onto if $X$ is path-connected, and that $\Phi([f]) = \Phi([g])$ if and only if 
  $[f]$ and $[g]$ are conjugate in $\pi_1(X, x_0)$.
  Hence $\Phi$ is a one-to-one correspondence between $[S^1, X]$ and $\pi_1(X)$ if $X$ is path-connected.  

\end{taggedexercise}

\begin{solution}
  
  \textit{Surjectivity}. Let $[h] \in [S^1, X]$ be a homotopy class of maps $S^1 \to X$. 
  Since $X$ is path-connected, certainly there is some $h \in [h]$ with $h(t) = x_0$ for some $t \in I$.
  Reperamaterize so that $h(0) = h(1) = x_0$.
  Then there must be some $f \in [f] \in \pi_1(X, x_0)$ s.t. $h$ is homotopic to $f$, hence $\Phi([f]) = [h]$, and so
  $\Phi$ is onto.

  \textit{Injectivity}. ($\impliedby$) Let $[f]$ and $[g]$ be conjugate in $\pi_1(X, x_0)$.
  Then there exists a homotopy $H$ mapping $f \to g$ for any $f \in [f]$ and $g \in [g]$.


\end{solution}

% Exercise 1.7
\begin{taggedexercise}[\textcolor{red}{TODO}]
  
\end{taggedexercise}

% Exercise 1.8
\begin{taggedexercise}[\textcolor{red}{TODO}]
  
\end{taggedexercise}

% Exercise 1.9
\begin{taggedexercise}[\textcolor{red}{TODO}]
  
\end{taggedexercise}

% Exercise 1.10
\begin{taggedexercise}[\textcolor{red}{TODO}]
  
\end{taggedexercise}

% Exercise 1.11
\begin{taggedexercise}[\textcolor{red}{TODO}]
  
\end{taggedexercise}

% Exercise 1.12
\begin{taggedexercise}[\textcolor{red}{TODO}]
  
\end{taggedexercise}

% Exercise 1.13
\begin{taggedexercise}[\textcolor{red}{TODO}]
  
\end{taggedexercise}

% Exercise 1.14
\begin{taggedexercise}[\textcolor{red}{TODO}]
  
\end{taggedexercise}

% Exercise 1.15
\begin{taggedexercise}[\textcolor{red}{TODO}]
  
\end{taggedexercise}

% Exercise 1.16
\begin{taggedexercise}[\textcolor{red}{TODO}]

  Show that there are no retractions $r : X \to A$ in the following cases: 
  \begin{enumerate}
    \item $X = \R^3$ and $A$ any subspace homeomorphic to $S^1$.
    \item $X = S^1 \times D^2$ and $A$ its boundary torus $S^1 \times S^1$.
    \item $X = S^1 \times D^2$ and $A$ the circle shown in the figure.
    \item 
  \end{enumerate}

\end{taggedexercise}
\begin{solution}
  
  (1). The fundamental group of the subspace of a retraction is a subgroup of the space. 
  $\pi_1(\R^3)$ is trivial, and $\pi_1(S^1) = \Z$, and so no retraction can exist as obviously 
  $\Z$ is not a subgroup of the trivial group.

  (2) The fundamental group of the subspace of a retraction is a subgroup of the space. 
  $\pi_1(S^1 \times D^2) = \Z$ as $S^1 \times D^2$ deformation retracts to $S^1$, whereas $\pi_1(S^1 \times S^1) = \Z \times \Z$.
  $\Z \times \Z$ cannot be a subgroup of $\Z$, hence the result.  

  (3) 

\end{solution}

% Exercise 1.17
\begin{taggedexercise}[\textcolor{red}{TODO}]
  
\end{taggedexercise}

% Exercise 1.18
\begin{taggedexercise}[\textcolor{red}{TODO}]
  
\end{taggedexercise}

% Exercise 1.19
\begin{taggedexercise}[\textcolor{red}{TODO}]
  
\end{taggedexercise}

% Exercise 1.20
\begin{taggedexercise}[\textcolor{red}{TODO}]
  
\end{taggedexercise}

% Exercise 1.21
\begin{taggedexercise}[\textcolor{red}{TODO}]
  
\end{taggedexercise}

% Exercise 1.22
\begin{taggedexercise}[\textcolor{red}{TODO}]
  
\end{taggedexercise}

% Exercise 1.23
\begin{taggedexercise}[\textcolor{red}{TODO}]
  
\end{taggedexercise}

\end{document}

