%==============================================================================
%  Algebraic Topology - Exercises
%  Initial LaTeX Commit
%==============================================================================

\documentclass[12pt]{book}

%------------------------------------------------------------------------------
%   Packages
%------------------------------------------------------------------------------
\usepackage[utf8]{inputenc}
\usepackage[T1]{fontenc}
\usepackage{lmodern}

\usepackage{tikz}
\usepackage{tikz-cd}
\usetikzlibrary{calc,3d}
\usetikzlibrary{decorations.markings}
\usepackage{animate}

\usepackage{amsmath,amssymb,amsthm}
\usepackage{mathtools}    % For extra math features (e.g., \DeclareMathOperator)
\usepackage{geometry}
\geometry{margin=1in}

\usepackage{xparse}
\usepackage{xcolor}

%------------------------------------------------------------------------------
%   Custom Commands for Algebraic Topology
%------------------------------------------------------------------------------
% Reduced homology, e.g. \rH_n(X)
\newcommand{\rH}{\widetilde{H}}

% Common functors in algebraic topology
\DeclareMathOperator{\Hom}{Hom}
\DeclareMathOperator{\Ext}{Ext}
\DeclareMathOperator{\Tor}{Tor}
\DeclareMathOperator{\colim}{colim}
\DeclareMathOperator{\hocolim}{hocolim}
\DeclareMathOperator{\holim}{holim}

% Smash and wedge products, etc.
\newcommand{\wedgeprod}{\wedge}
\newcommand{\wedgesum}{\vee}

% Often we override \Im, \Re, \ker, etc. for nicer formatting:
\renewcommand{\Im}{\operatorname{Im}}
\renewcommand{\Re}{\operatorname{Re}}
\DeclareMathOperator{\Ker}{Ker}

% Common Fields
\newcommand{\R}{\mathbb{R}}        % Real numbers
\newcommand{\C}{\mathbb{C}}        % Complex numbers
\newcommand{\HH}{\mathbb{H}}        % Quarternions
\newcommand{\Q}{\mathbb{Q}}        % Rational numbers
\newcommand{\Z}{\mathbb{Z}}        % Integers
\newcommand{\N}{\mathbb{N}}        % Natural numbers
\newcommand{\K}{\mathbb{K}}        % Arbitrary field

%------------------------------------------------------------------------------
%   Define 'taggedexercise' environment
%------------------------------------------------------------------------------
% We want each exercise to appear in the ToC at subsection level
\NewDocumentEnvironment{taggedexercise}{O{}}
{%
  \refstepcounter{exercise}%
  \par\noindent
  \textbf{Exercise \theexercise}%
  \if\relax\detokenize{#1}\relax
  \else
    \textbf{\ [#1]}%
  \fi
  \quad
  \addcontentsline{toc}{subsection}{%
    Exercise \theexercise
    \if\relax\detokenize{#1}\relax
    \else
      \ [#1]%
    \fi
  }
}{%
  \par
}

%------------------------------------------------------------------------------
%   Theorem-Like Environments
%------------------------------------------------------------------------------
\theoremstyle{definition}
\newtheorem{exercise}{Exercise}[chapter]

% Custom solution environment
\newenvironment{solution}
{%
  \par\noindent\textbf{Solution.}\quad
}
{%
  \qed\par
}

%------------------------------------------------------------------------------
%   Metadata
%------------------------------------------------------------------------------
\title{Exercises from \textit{Algebraic Topology} \\
       by Allen Hatcher}
\author{Tyler Jensen | tyjensen222@gmail.com}
\date{\today}

%------------------------------------------------------------------------------
%   Document
%------------------------------------------------------------------------------
\begin{document}

\frontmatter
\maketitle
\tableofcontents

\mainmatter

%------------------------------------------------------------------------------
%  START CHAPTERS AT 0
%------------------------------------------------------------------------------
\setcounter{chapter}{-1}

%==============================================================================
\chapter{Some Underlying Geometric Notions}
%==============================================================================


% Exercise 0.1
\begin{taggedexercise}[\textcolor{green}{Complete}]
  Construct an explicit deformation retraction of the torus with one point deleted onto a graph of two circles intersecting at a point, namely, longitude and meridian circles of the torus.
\end{taggedexercise}
\begin{solution}
  Consider the fundamental square of the torus with a point removed:
  \begin{center}
    \begin{tikzpicture}[
      scale=2,  % (1) Scale up
      arrow inside/.style = {
        postaction={decorate},
        decoration={markings, mark=at position 0.5 with {\arrow{stealth}}}
      }
    ]
    
      %--- Draw the square with edges labeled a and b (with arrows in the middle)
      \draw[arrow inside] (0,0) -- node [below] {$a$} (1,0);
      \draw[arrow inside] (0,1) -- node [above] {$a$} (1,1);
      \draw[arrow inside] (0,0) -- node [left]  {$b$} (0,1);
      \draw[arrow inside] (1,0) -- node [right] {$b$} (1,1);
    
      %--- (2) Small circle in the center to mark the removed point
      \draw (0.5,0.5) circle (0.03);
    \end{tikzpicture}
  \end{center}
  where the deformation retract of the removed point retaracts to the boundary of the square; identifying the sides gives a bouquet of two circles:
  \begin{center}
    \begin{tikzpicture}[scale=2.0]

      % We define a style "side arrow" that places
      % a large arrow tip at a specified position around the circle.
      % - 'mark=at position <number>' controls where along the path [0..1].
      % - 'scale=1.8' (for example) enlarges the arrowhead.
      \tikzset{
        side arrow/.style={
          postaction={decorate},
          decoration={
            markings,
            mark=at position #1 with {\arrow[scale=1.8]{stealth}}
          }
        }
      }
    
      % Circle 'a'
      %  -- Center: (-0.5,0), Radius: 0.5
      %  -- Place arrow at 0.25 (1/4 around the circle),
      %     so the arrow is off to the side.
      \draw[side arrow=0.25]
           (-0.5,0) circle (0.5);
      % Label 'a' near top
      \node at (-0.5,0.3) {$a$};
    
      % Circle 'b'
      %  -- Center: (0.5,0), Radius: 0.5
      %  -- Put arrow at 0.75 to shift it to the opposite side
      \draw[side arrow=0.75]
           (0.5,0) circle (0.5);
      % Label 'b' near top
      \node at (0.5,-0.3) {$b$};
    
      % Wedge point
      \fill (0,0) circle (0.02);
    
    \end{tikzpicture}
  \end{center}
  where the point at the intersection of the circles is the identification of the four corners.

  Alternatively, let $p \in T$ be the removed point.
  Choose your favorite open set $U$ about $p$.
  Retract $U$ to the meridian, so that $T$ is now missing a longitudinal strip, contracted to  a line spanning the meridial width of $U$.
  Then there are two open copies of $S^1$ seperated by this width; contract them along the meridian in the opposite direction of eachother.
  Then, exactly two copies of $S^1$ are left, connected by a single point.
\end{solution}

\begin{taggedexercise}[\textcolor{red}{TODO}]
  Construct an explicit deformation retract from $\R^n - \{0\}$ to $S^{n-1}$.
\end{taggedexercise}
\begin{solution}
  Define a map 
  \[h: \left(\R^n - \{0\}\right) \times I \to \R^n - \{0\}\]
  via 
  \[h(x, t):= (1-t)x + \frac{tx}{||x||}\]
  where $||\cdot ||$ is the Euclidean norm in $\R^n$.
  Then, $h(x, 0) = x$ and $h(1) = x/||x||$, retracting each point to a point on $S^{n-1}$.

\end{solution}

%==============================================================================
\chapter{Fundamental Group}
%==============================================================================

... content ...

\end{document}

